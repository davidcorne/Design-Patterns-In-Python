\documentclass[12pt, oneside]{book} %,oneside
\setlength{\topmargin}{-.5in} % top margin is .5 in
\setlength{\oddsidemargin}{0in} % left margin is 1 in on right pages
\setlength{\evensidemargin}{0in} % same for left pages, 2-sided document
\setlength{\textwidth}{6.5in} % leaves 1 in for right margin
\setlength{\textheight}{9in}
\linespread{1.6}

% gets hyperling references
\usepackage{hyperref}

%\usepackage{amsmath}
%\usepackage{amsfonts}
%\usepackage{amssymb}

% packages for the front page
\usepackage{graphicx}
\newcommand{\HRule}{\rule{\linewidth}{0.5mm}} %Create Hrule command for title page

% colour for prettiness
\usepackage{color}
\usepackage[usenames, dvipsnames, table]{xcolor}

% code syntax higlighting
\usepackage{listings}
\lstset{ %
  language = Python,                % choose the language of the code
  basicstyle = \scriptsize,         % the size of the fonts that are used for the code
  % line number commands
  numbers = left,                   % where to put the line-numbers
  numberstyle = \scriptsize,        % the size of the fonts that are used for the line-numbers
  stepnumber = 1,                   % the step between two line-numbers. If it is 1 each line will be numbered
  numbersep = 5pt,                  % how far the line-numbers are from the code
  % generic options
  backgroundcolor = \color{white},  % choose the background color.
  showspaces = false,               % show spaces adding particular underscores
  showstringspaces = false,         % underline spaces within strings
  showtabs = true,                  % show tabs within strings adding particular underscores
  frame = single,                   % adds a frame around the code
  tabsize = 2,                      % sets default tabsize to 2 spaces
  captionpos = b,                   % sets the caption-position to bottom
  breaklines = true,                % sets automatic line breaking
  breakatwhitespace = false,        % sets if automatic breaks should only happen at whitespace
  escapeinside = {\%*}{*)},         % if you want to add a comment within your code
  % syntax highlighting colours
  keywordstyle = \color{Purple},
  identifierstyle = \color{RoyalBlue},
  commentstyle = \color{OliveGreen},
  stringstyle = \color{Red}
}


% package for making the chapters big and blue
\usepackage{quotchap}                       % change the \chapter command to give alrge numbers right alligned
\colorlet{chaptergrey}{RoyalBlue}           % make the number blue
\renewcommand*\sectfont{\color{RoyalBlue}}  % make the writing blue

% reduce space after sections
\usepackage{titlesec}
\titlespacing\section{0pt}{8pt plus 2pt minus 2pt}{-2\parskip}

\newcommand{\tmpsection}[1]{}
\let\tmpsection=\section
\renewcommand{\section}[1]{
  { 
    \samepage{
      \vspace{10 mm} \linespread{0.75} \color{RoyalBlue}{
        \noindent \HRule \tmpsection{#1} \noindent \HRule
      }
    }
  }
}


% design pattern summary command
\newcommand{\patternSummaryGoalMotProCon}[4]{
  \begin{center}
    \begin{tabular}{|l|p{12cm} |}
      \hline
      \color{RoyalBlue}{Goal} & #1
      \\
      \hline
      \color{RoyalBlue}{Motivation} & #2
      \\ 
      \hline
      \color{RoyalBlue}{Pros} & #3
      \\ 
      \hline
      \color{RoyalBlue}{Cons} & #4
      \\ 
      \hline
    \end{tabular}
  \end{center}
}

\begin{document}

% % title page
% \thispagestyle{empty}
% % Upper part of the page
% \begin{center}
%   \includegraphics[width=0.25\textwidth]{Python_Logo.png}\\[1cm]
% \end{center} 
% % Title
% \HRule \\[0.4cm]
%        {\Huge \bfseries \color{RoyalBlue}Design Patterns in Python} \\[0.4cm]
%        {\huge or How I Learned to Stop Worrying and Love The Gang of Four} \\[0.4cm]
% \HRule \\[1.5cm]
% 
% % Author
% \begin{minipage}{0.4\textwidth}
%   \begin{flushleft}
%     \Large David Corne
%   \end{flushleft}
% \end{minipage}
% 
% \vfill
% % Bottom of the page
% % {\large August 2012.}
% 
% \frontmatter
% \tableofcontents
% 
% \chapter{Acknowledgments}
% \mainmatter

\chapter{Introduction}
There are many places this is usful, there have been a few tries to do this.

many patterns given to you by the language

\section{Gang of Four}
This influential book .......
\cite{python-patterns-site} contains a few patterns and I have drawn from those


\chapter{Creational}
\label{creational}

\section{Abstract Factory}
% table of information for the start of a section
\patternSummaryGoalMotProCon{}{}{}{}
Some kind of intro



as said in chapter \ref{creational} in book \cite{GOF}

\section{Builder}
% table of information for the start of a section
\patternSummaryGoalMotProCon{}{}{}{}

\section{Factory Method}
% table of information for the start of a section
\patternSummaryGoalMotProCon{}{}{}{}

\section{Prototype}
% table of information for the start of a section
\patternSummaryGoalMotProCon{}{}{}{}

\section{Singleton}
% table of information for the start of a section
\patternSummaryGoalMotProCon{
For the class to only be able to have one instance.
}
{
Structures like global variables or settings.
}
{
\begin{itemize}
\item The data acess is very controllable, and extensable.
\end{itemize}
}
{
\begin{itemize}
\item Introduces a global state into the program, with all the pitfalls associated with that.
\item 
\end{itemize}
}

This is certainly the most controversial pattern on this list. 

In C++ you have to ensure that a new instance of the singleton cannot be made. This is typically done by protecting the constructors and making a static \lstinline[language = c++]{Instance()} function which returns the current instance, or makes a new one if it is appropriate. \\

This is another easy pattern to implement in Python. It is a case of overriding the \lstinline{__new__()} function. This is a static method for all classes which is called on construction. This function is called with an argument of the class type requested and returns a new instance of that class. So to implement the singleton pattern it is a case of making sure that the \lstinline{__new__()} method returns the global instance. This can be done with the following code:
\begin{lstlisting}
    def __new__(new_singleton):
        """Override the __new__ method so that it is a singleton."""
        if not new_singleton.instance:
            new_singleton.instance = super(Singleton, new_singleton).__new__(new_singleton)
            print("An actual new instance of Singleton made.")
            print("")
        return new_singleton.instance
\end{lstlisting}
Here the class name is Singleton and instance is a member variable initialised to \lstinline{None}.

\chapter{Structural}

\section{Adapter}
% table of information for the start of a section
\patternSummaryGoalMotProCon{}{}{}{}

\section{Bridge}
% table of information for the start of a section
\patternSummaryGoalMotProCon{}{}{}{}

\section{Composite}
% table of information for the start of a section
\patternSummaryGoalMotProCon{}{}{}{}

\section{Decorator}
% table of information for the start of a section
\patternSummaryGoalMotProCon{}{}{}{}

\section{Facade}
% table of information for the start of a section
\patternSummaryGoalMotProCon{}{}{}{}

\section{Flyweight}
% table of information for the start of a section
\patternSummaryGoalMotProCon{}{}{}{}

\section{Proxy}
% table of information for the start of a section
\patternSummaryGoalMotProCon{}{}{}{}


\chapter{Behavioural}

\section{Chain of Responsibility}
% table of information for the start of a section
\patternSummaryGoalMotProCon{}{}{}{}

\section{Command}
% table of information for the start of a section
\patternSummaryGoalMotProCon{}{}{}{}

\section{Interpreter}
% table of information for the start of a section
\patternSummaryGoalMotProCon{}{}{}{}

\section{Iterator}
% table of information for the start of a section
\patternSummaryGoalMotProCon{}{}{}{}
This is a pattern which falls into the catagory of included in Python already. This is because anything which can be iterated over is able to be. This goes for built in classes but if you want to make an iterable class you need to supply two methods; \lstinline{__iter__()} and \lstinline{next()}. These two methods can be called to explicitly make a loop or they will be called in a for loop. For instance in the example code there are the following lines:
% [firstline = 35]
\begin{lstlisting}
    for day in it:
        print(day)
\end{lstlisting}
Which give exactly the same result as:
%\lstlisting[firstline = 35]{
%thing
% }

\section{Mediator}
% table of information for the start of a section
\patternSummaryGoalMotProCon{}{}{}{}

\section{Memento}
% table of information for the start of a section
\patternSummaryGoalMotProCon{}{}{}{}

\section{Observer}
% table of information for the start of a section
\patternSummaryGoalMotProCon{}{}{}{}

\section{State}
% table of information for the start of a section
\patternSummaryGoalMotProCon{}{}{}{}

\section{Strategy}
% table of information for the start of a section
\patternSummaryGoalMotProCon{}{}{}{}

\section{Template Method}
% table of information for the start of a section
\patternSummaryGoalMotProCon{}{}{}{}

\section{Visitor}
% table of information for the start of a section
\patternSummaryGoalMotProCon{}{}{}{}

\chapter{Non GOF Patterns/GUI Patterns}
Use of Tkinter.

\section{Model View Presenter}
% table of information for the start of a section
\patternSummaryGoalMotProCon{}{}{}{}

http://stackoverflow.com/questions/2056/what-are-mvp-and-mvc-and-what-is-the-difference

\section{Model View Controller}
% table of information for the start of a section
\patternSummaryGoalMotProCon{}{}{}{}

\section{Model View View-Model}
% table of information for the start of a section
\patternSummaryGoalMotProCon{}{}{}{}

\chapter{Summary}

% \part*{Appendicies}
% \addcontentsline{toc}{part}{Appendicies}
% \appendix
% \chapter{Creational Code Examples}
% 
% \section{Abstract Factory}
% \lstinputlisting{Abstract_Factory.py}
% 
% \section{Builder}
% \lstinputlisting{Builder.py}
% \section{Factory Method}
% \lstinputlisting{Factory_Method.py}
% \section{Prototype}
% \lstinputlisting{Prototype.py}
% \section{Singleton}
% \lstinputlisting{Singleton.py}
% 
% \chapter{Structural Code Examples}
% 
% \section{Adapter}
% \lstinputlisting{Adapter.py}
% \section{Bridge}
% \lstinputlisting{Bridge.py}
% \section{Composite}
% \lstinputlisting{Composite.py}
% \section{Decorator}
% \lstinputlisting{Decorator.py}
% \section{Facade}
% \lstinputlisting{Facade.py}
% \section{Flyweight}
% \lstinputlisting{Flyweight.py}
% \section{Proxy}
% \lstinputlisting{Proxy.py}
% 
% \chapter{Behavioural Code Examples}
% 
% \section{Chain of Responsibility}
% \lstinputlisting{Chain_of_Responsibility.py}
% \section{Command}
% \lstinputlisting{Command.py}
% \section{Interpreter}
% \lstinputlisting{Interpreter.py}
% \section{Iterator}
% \lstinputlisting{Iterator.py}
% \section{Mediator}
% \lstinputlisting{Mediator.py}
% \section{Memento}
% \lstinputlisting{Memento.py}
% \section{Observer}
% \lstinputlisting{Observer.py}
% \section{State}
% \lstinputlisting{State.py}
% \section{Strategy}
% \lstinputlisting{Strategy.py}
% \section{Template Method}
% \lstinputlisting{Template_Method.py}
% \section{Visitor}
% \lstinputlisting{Visitor.py}
% 
% \chapter{Non GOF Patterns/GUI Patterns Code Examples}
% \section{Model View Presenter}
% \lstinputlisting{Model_View_Presenter.py}
% \section{Model View Controller}
% \lstinputlisting{Model_View_Controller.py}
% \section{Model View View-Model}
% \lstinputlisting{Model_View_View-Model.py}
% 
% \begin{thebibliography}{99}
% \bibitem{GOF} Erich Gamma, Richard Helm, Ralph Johnson and John Vlissides, \emph{Design Patterns: Elements of Reusable Object-Oriented Software} (: Addison-Wesley Professional, January 15, 1995)
% \bibitem{python-patterns-site} http://ginstrom.com/scribbles/2007/10/08/design-patterns-python-style
% \end{thebibliography}
\end{document}



